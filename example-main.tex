\documentclass[11pt]{article}
\usepackage{xltxtra}
\usepackage{bookmark}
\usepackage{hyperref}
\hypersetup{hidelinks}
\usepackage{url}
\urlstyle{tt}
\usepackage{multicol}
\usepackage{xcolor}
\usepackage{calc}
\usepackage{graphicx}
\usepackage{tikz}
\usetikzlibrary{calc}
\usepackage{fontspec}
\usepackage{xeCJK}
\usepackage{relsize}
\usepackage{xspace}
\usepackage{fontawesome}
\usepackage{titlesec}
\usepackage{enumitem}
\usepackage{siunitx}
\usepackage{amssymb}
\usepackage{tabularx}
\usepackage{multicol}
\usepackage{fontspec}


% 一些小设置,参考自https://github.com/LeyuDame/BNUCV/tree/main
\CJKsetecglue{}							            % 取消中文字符与数字之间的间隔
\protected\def\Cpp{{C\nolinebreak[4]\hspace{-.05em}\raisebox{.28ex}{\relsize{-1}++}}\xspace}	% 这是个更好看的C++写法,你直接写C++的话,+号会很大,可以使用\Cpp来代替
\setlength{\parindent}{0pt}							% 取消全局段落缩进
\pagenumbering{gobble}								% 取消页码显示
%\setlist{noitemsep}									% 禁用列表中项目之间的额外垂直间距,但保留列表周围的间距
%\setlist{nosep}										% 禁用列表中项目之间的额外垂直间距及列表周围的间距
\setlist[itemize]{topsep=0em, leftmargin=*}		% 增加了itemize顶部间距
\setlist[enumerate]{topsep=0em, leftmargin=*}	% 增加了enumerate顶部间距

\titleformat{\section}					    % 将原标题前面的数字取消了
  {\LARGE\bfseries\raggedright} 		      % 字体改为LARGE,bold,左对齐
  {}{0em}                      			  % 可用于添加全局标题前缀
  {}                           			  % 可用于添加代码
  [{\color{wustgreen}\titlerule}]            % 标题下方加一条线
\titlespacing*{\section}{0cm}{*1.2}{*1.2}	% 标题左边留白,上方1.2倍,下方1.2倍

\titleformat{\subsection}				    % 将原二级标题前面的数字取消了
  {\large\bfseries\raggedright} 		      % 字体改为large,bold,左对齐
  {}{0em}                      			  % 可用于添加全局二级标题前缀
  {}                           			  % 可用于添加代码
  []
\titlespacing*{\subsection}{0cm}{*1.2}{*1.2}% 二级标题左边留白,上方1.2倍,下方1.2倍

% 页面大小与页边距,按需求调整
\usepackage[
	a4paper,
	left=1.2cm,
	right=1.2cm,
	top=1.5cm,
	bottom=1cm,
	nohead
]{geometry}

% 中文字符间距
\renewcommand{\CJKglue}{\hskip 0.05em}

% 英文字体
\setmainfont[
    Path=fonts/,
    Extension=.ttf,
    BoldFont=* Bold,
]{Microsoft Yahei}
% 中文字体
\setCJKmainfont[
    Path=fonts/,
    Extension=.ttf,
    BoldFont=* Bold,
]{Microsoft Yahei}

% 主题色
\definecolor{wustgreen}{RGB}{0, 128, 0}

% 这里把表格的行间距调成1.2倍了
\renewcommand{\arraystretch}{1.2}
% 这里把正文的行间距调成1.2倍了
\linespread{1.2}

%%%%%%%%%%%%%%%%%%%%%%%%%%%%%%%%%%%%%%%%%%%%%%%%%%%%%%%%%%%%%%%%%%%%%%%%%%%%%%%%%%%%%%%%%%%%%%%%%%%%%%%%%%%%%%%%%%%%%%%%%%%%%%%%%%%%%%%%%%%%%%%%%%
%    !!!!!!!! 记得改这里 !!!!!!!!
%%%%%%%%%%%%%%%%%%%%%%%%%%%%%%%%%%%%%%%%%%%%%%%%%%%%%%%%%%%%%%%%%%%%%%%%%%%%%%%%%%%%%%%%%%%%%%%%%%%%%%%%%%%%%%%%%%%%%%%%%%%%%%%%%%%%%%%%%%%%%%%%%%
% 学院
\newcommand{\school}{理学院 | School of Mathematics}
% 也可以不写英语
%\newcommand{\school}{电子信息学院}
% 联系方式
\newcommand{\contact}
{
    \small              % 换了更小的字号
    % \footnotesize       % 这比上面的小一号
    \scriptsize         % 这比上面的再小一号
    \textcolor{white}
    {
        \faEnvelope \quad \href{mailto:xxx@gmail.com}{xxx@gmail.com}    % 邮箱,前面的超链接可以直达邮箱软件
        \hspace{4em}    % 这里可以调间距
        \faWechat \quad Reyussend               % 微信
        \hspace{4em}    % 这里可以调间距
        \faPhone \quad 10086                  % 手机号
        \hspace{4em}    % 这里可以调间距
        \faGithub \quad \href{https://github.com/xxxx}{https://github.com/xxxx}         % github
    }
}



%%%%%%%%%%%%%%%%%%%%%%%%%%%%%%%%%%%%%%%%%%%%%%%%%%%%%%%%%%%%%%%%%%%%%%%%%%%%%%%%%%%%%%%%%%%%%%%%%%%%%%%%%%%%%%%%%%%%%%%%%%%%%%%%%%%%%%%%%%%%%%%%%%
%    !!!!!!!! 这里开始就是正文了 !!!!!!!!
%%%%%%%%%%%%%%%%%%%%%%%%%%%%%%%%%%%%%%%%%%%%%%%%%%%%%%%%%%%%%%%%%%%%%%%%%%%%%%%%%%%%%%%%%%%%%%%%%%%%%%%%%%%%%%%%%%%%%%%%%%%%%%%%%%%%%%%%%%%%%%%%%%
\begin{document}
	% 如果有多页简历,请把页眉页脚和背景复制粘贴到第二页的内容之前
	% 页眉,校徽,学院名
	\begin{tikzpicture}[remember picture, overlay]
		\node[anchor=north, inner sep=0pt](header) at (current page.north){
			\includegraphics[width=\paperwidth]{images/header.png}
		};
		\node[anchor=west](school_logo) at (header.west){
    \hspace{0.5cm}
    \includegraphics[width=0.25\textwidth]{images/wust-logo.png}
};
		\node[anchor=east](school_name) at(header.east){
			\textcolor{white}{\textbf{\school}}
			\hspace{0.5cm}
		};
	\end{tikzpicture}
	\vspace{-4em}

	% 页脚,联系方式
	\begin{tikzpicture}[remember picture, overlay]
		\node[anchor=south, inner sep=0pt](footer) at (current page.south){
			\includegraphics[width=\paperwidth]{images/footer.png}
		};
        % 联系方式
        \node[anchor=center] at(footer.center){\contact};
	\end{tikzpicture}
	
	% 背景
	\begin{tikzpicture}[remember picture, overlay]
		\node[opacity=0.1] at(current page.center){
			\includegraphics[width=0.7\paperwidth, keepaspectratio]{images/wust_logo_big.png}
		};
	\end{tikzpicture}

	% 个人信息
    \begin{figure}[h]
        % 左半边,信息,比例占行宽87%,可以自己调
        \begin{minipage}{0.87\textwidth}
            \section{\makebox[\widthof{\faUser}][c]{\color{wustgreen}{\faUser}}\quad 个人信息}
            \begin{tabularx}{\linewidth}{p{\widthof{出生日期:}}Xp{\widthof{政治面貌:}}X}
                姓名: & sueiran & 性别: & 男 \\
                出生日期: & 1978年11月18日 & 政治面貌: & 良民 \\
                    
                %% 想多加几行的话,就按上面的格式自行补充
                %% 想加粗的话\textbf{}
                %% 想多加几列的话,把\begin{tabularx}{\textwidth}{这里}的内容改一下,可以自己搜一下tabularx怎么用,也可以问gpt/文心一言/讯飞。
            \end{tabularx}
        \end{minipage}
        % 右半边,照片,比例占行宽12%,可以自己调
        % images/example_avatar.png 替换成你证件照的路径。
        \begin{minipage}{0.12\textwidth}
            \includegraphics[width=\linewidth]{images/sueiran.jpg}
        \end{minipage}
        % 尽量留至少1%的间距,不然会换行
    \end{figure}


	% 教育背景
    	% \faGraduationCap这类\fa开头的都是font awesome里的logo,想换成其他logo的话,可以看一下附带的fontawsome.pdf,自行替换。
		% \section{\makebox[\widthof{     这里!    }][c]{\color{NPU_Blue}{     和这里!    }}\quad 标题}
	\section{\makebox[\widthof{\faGraduationCap}][c]{\color{wustgreen}{\faGraduationCap}}\quad 教育背景}
	\vspace{-1em}
    \begin{table}[h!]
        \begin{tabularx}{\textwidth}{XXp{\widthof{2021年 -- 预计2025年7月毕业}}}
            武汉科技大学理学院 & 统计学 & 2022年 -- 预计2026年6月毕业\\
            \textbf{GPA: 2.713/4} & \textbf{GPA排名: 75/76} & \textbf{综测排名: 74/76} \\
            % 这里哪个高就加粗哪个,哪个不想放就留白 =w= 比如你综测高GPA低,你就只放综测和综测排名就好。
        \end{tabularx}
    \end{table}

    % 项目经历(找导师一般都看中这个),可以改成“科研经历”
        % \faGears 这是齿轮,适合机械类,我电信的也喜欢齿轮,就用这个了
        % \faFlask 这是烧瓶,适合生化类
        % \faLaptop 这是个笔记本电脑,适合计算机类
        % \faUsers 这是三个人,适合商科
   \section{\makebox[\widthof{\faLaptop}][c]{\color{wustgreen}{\faLaptop}}\quad 项目经历}
    \vspace{0.5em}
    
    % 第一个项目:侧重于扩散模型的采样加速或模型蒸馏(硬核算法)
    \subsection{Fast-Diffusion: 基于知识蒸馏的高效率扩散模型加速算法研究\hfill ECCV Oral}
    
    \textbf{第一负责人} \hfill 2022年7月-2023年7月
    
    主要负责扩散模型在推理阶段的\textbf{步数压缩与采样加速}。针对 DDIM 采样速度慢的问题,提出了一种基于\textbf{渐进式蒸馏(Progressive Distillation)}的改进方案。通过引入 \textbf{Consistency Models(一致性模型)} 理论,成功将生成步数从 50 步缩减至 \textbf{1-4 步},且保持了极高的 \textbf{FID 指标}。该项目最终实现了在移动端实时生成高质量图像,并获得校级优秀科研成果奖。
    
    \vspace{1em}
    % 第二个项目:侧重于多模态与精准控制(ControlNet/Adapter 逻辑)
    \subsection{Multi-Modal ControlNet: 基于多模态引导的图像可控生成研究 \hfill ICCV}

    \textbf{学生独作} \hfill 2023年12月-2024年5月
    
    \textbf{独立负责模型架构设计、大规模预训练及定量消融实验}。重点探索了将\textbf{深度图、语义分割图与文本提示词}相结合的多模态控制方法。通过在 Stable Diffusion 骨干网络中嵌入自研的 \textbf{Lightweight-Adapter} 模块,实现了在不微调原模型权重的前提下,对生成空间结构的精确控制。实验结果表明,该模型在 \textbf{COCO 数据集} 的人机评价中显著优于传统的 ControlNet,推理显存占用降低了 \textbf{30\%}。

    \vspace{1em}
    % 第三个项目:侧重于细分领域应用(如医疗影像增强或视频生成)
    \subsection{DiffMedical: 基于生成式扩散模型的医学影像超分辨率重建\hfill CVPR}
        
    \textbf{第一作者} \hfill 2024年6月-至今
    
    针对医学成像(MRI/CT)中存在的噪声大、分辨率低等痛点,研发了一套基于\textbf{条件扩散模型(Conditional Diffusion)}的增强系统。利用 \textbf{Latent Diffusion} 空间下的先验知识进行补全,解决了传统 GAN 模型在医学图像生成中容易出现的\textbf{模式崩塌(Mode Collapse)}问题。通过引入 \textbf{Cross-Attention} 机制融合病灶解剖学特征,将图像的 \textbf{PSNR 指标}提升了 15\%,有效辅助了放射科医师对早期微小病灶的识别。

    \vspace{1em}
    % 第四个项目:侧重于落地实践或比赛(对应大国工匠,改为底层优化)
    \subsection{高性能扩散模型推理引擎开发\hfill 工程项目-已完结}
        
    \textbf{主要技术负责人} \hfill 2023年上半年
    
    负责扩散模型在 \textbf{TensorRT} 框架下的底层算子优化与量化部署。针对 Transformer 结构的 Attention 层进行了 \textbf{FlashAttention} 融合优化,并将模型权重成功量化至 \textbf{INT8} 精度而不损失生成多样性。通过自研的显存管理策略,实现了在 8G 显存显卡上进行 \textbf{1024x1024 像素}级别的超清图像生成,极大提升了模型在工业级场景下的吞吐量。

    % 终于用完一页了,加一页展示怎么加页眉页脚
    \newpage

    % 宝子,请加在这里
	% 页眉,校徽,学院名
	\begin{tikzpicture}[remember picture, overlay]
		\node[anchor=north, inner sep=0pt](header) at (current page.north){
			\includegraphics[width=\paperwidth]{images/header.png}
		};
		\node[anchor=west](school_logo) at (header.west){
			\hspace{0.5cm}
			\includegraphics[width=0.25\textwidth]{images/wust-logo.png}
		};
		\node[anchor=east](school_name) at(header.east){
			\textcolor{white}{\textbf{\school}}
			\hspace{0.5cm}
		};
	\end{tikzpicture}
	\vspace{-4em}

	% 页脚,联系方式
	\begin{tikzpicture}[remember picture, overlay]
		\node[anchor = south, inner sep=0pt] at (current page.south){
			\includegraphics[width=\paperwidth]{images/footer.png}
		};
        % 联系方式
        \node[anchor=center] at(footer.center){\contact};
	\end{tikzpicture}
	
	% 背景
	\begin{tikzpicture}[remember picture, overlay]
		\node[opacity=0.1] at(current page.center){
			\includegraphics[width=0.7\paperwidth, keepaspectratio]{images/wust_logo_big.png}
		};
	\end{tikzpicture}

    % 竞赛经历(找导师也可能看中这个,因为代表了一定实践能力,但是尽量对口吧,不要运动会都写进去了)
    \section{\makebox[\widthof{\faTrophy}][c]{\color{wustgreen}{\faTrophy}}\quad 竞赛经历}
    \vspace{-1em}
    \begin{table}[h!]
        \begin{tabularx}{\textwidth}{Xp{\widthof{第一负责人}}p{\widthof{国家级-特等奖}}p{\widthof{2023年12月}}}
            \textbf{第九届中国大学生电子竞技联赛 (UCEL)} & 队长/指挥 & 国家级-前5 & 2023年11月 \\
            \textbf{2023年英雄联盟全国高校联赛 (LCL)} & 核心首发 & 国家级-一等奖 & 2023年8月\\
            \textbf{2022年王者荣耀高校区域联赛} & 个人参赛 & 省级-冠军 & 2022年12月\\
            % 如果有更多比赛可以继续往这里加
        \end{tabularx}
    \end{table}

    % 技能特长,上面写很多的话,这里就随便写点,反正上面都看出来了。上面写的不多的话,这里着重强调你会什么。
    % 哦,你找工作的话,这里多写点,记得对口,可以\textbf{}加粗。
    % 这里能吹牛皮就吹牛皮,但是确保面试的时候别露馅就行。
    \section{\makebox[\widthof{\faWrench}][c]{\color{wustgreen}{\faWrench}}\quad 技能特长}
    \vspace{0.5em}
    \begin{itemize}
    \item \textbf{编程语言}:精通 \Cpp (CUDA)、Python、Verilog (硬件描述语言);熟悉汇编语言。
    \item \textbf{高性能计算}:熟练掌握 \textbf{CUDA} 编程模型,具备显卡底层算子优化、并行计算与内存管理经验。
    \item \textbf{深度学习框架}:深谙 \textbf{PyTorch / TensorRT} 推理后端优化;熟悉显卡架构下的 \textbf{FlashAttention} 及算子融合技术。
    \item \textbf{芯片设计与仿真}:熟练使用 \textbf{Vivado / Quartus} 进行 FPGA 原型验证;掌握逻辑综合与时序约束分析。
    \item \textbf{显卡架构与部署}:熟悉 NVIDIA GPU 架构(Ampere/Hopper),具备 \textbf{FP8/INT8 量化}、模型剪枝及移动端 NPU 适配经验。
    \item \textbf{底层驱动与开发}:精通嵌入式 Linux 环境,具备驱动开发、PCIe 通信协议及显存直接访问 (DMA) 调试经验。
    \item \textbf{算法前沿}:深入理解硬件友好的生成式 AI 算法,如 \textbf{Efficient Diffusion}、模型蒸馏、结构化剪枝等。
\end{itemize}

    % 所获荣誉(这个看你想不想写了)
    \section{\makebox[\widthof{\faStar}][c]{\color{wustgreen}{\faStar}}\quad 所获荣誉}
    \vspace{-1em}
    \begin{multicols}{2}
        \begin{itemize}
            \item 某年学业先进个人
            \item 某年某奖学金某等奖
            \item 某大使
            \item 某年某奖学金某等奖
            \item 某年优秀团员称号
            \item 某年某称号
        \end{itemize}
    \end{multicols}

    % 其他(也是看你想不想写)
    \section{\makebox[\widthof{\faInfo}][c]{\color{wustgreen}{\faInfo}}\quad 其他}
    \begin{itemize}
        \item 英语水平-CET6级800分
        \item 计算机九级证书
        \item 技术博客: xxx.xyz.com
        \item 教师资格证:xxx
        \item 普通话证书:丁级
        \item 文字排版:\LaTeX
    \end{itemize}

\end{document}
